\capitulo{6}{Anexo de sostenibilización curricular}

\section{Introducción}

La sostenibilidad se ha convertido en uno de los ejes fundamentales en el desarrollo de políticas públicas, actividades económicas y proyectos tecnológicos. En el ámbito agrario, esta importancia es aún mayor, debido a la estrecha relación existente entre la producción agrícola, el uso de los recursos naturales y la conservación del medio ambiente. En este contexto, el presente Trabajo Fin de Grado se alinea con los principios de sostenibilidad promovidos tanto a nivel europeo como nacional, especialmente a través de la Política Agraria Común (PAC).

Este anexo tiene como objetivo analizar y justificar cómo el proyecto desarrollado contribuye a la sostenibilidad desde una perspectiva ambiental, social y económica, así como su alineación con los Objetivos de Desarrollo Sostenible (ODS). La aplicación desarrollada no solo resuelve un problema técnico, sino que también aporta valor real al sector agrícola, facilitando la toma de decisiones responsables y sostenibles por parte de los agricultores.

\section{Contribución a la sostenibilidad ambiental}

El proyecto contribuye de forma directa a la sostenibilidad ambiental mediante la optimización de la gestión de cultivos y el cumplimiento de los criterios establecidos por la PAC en materia de buenas prácticas agrarias y medioambientales.

La aplicación permite analizar el histórico de cultivos de cada parcela y recinto, lo que facilita la correcta planificación de rotaciones de cultivo. Esta rotación es fundamental para mantener la fertilidad del suelo, reducir la aparición de plagas y enfermedades, y disminuir la necesidad de productos fitosanitarios. Además, el sistema fomenta el uso de especies mejorantes, como las leguminosas, que contribuyen a la fijación natural de nitrógeno en el suelo y reducen la dependencia de fertilizantes químicos.

Asimismo, la generación de propuestas de cultivo basadas en criterios PAC y ecorregímenes ayuda a que las explotaciones agrarias cumplan con prácticas respetuosas con el medio ambiente, como la diversificación de cultivos o la reducción de superficies improductivas innecesarias. Todo ello favorece un uso más racional y sostenible de la tierra agrícola.

\section{Impacto en la sostenibilidad social}

Desde el punto de vista social, el proyecto tiene un impacto significativo en la mejora de las condiciones de trabajo de los agricultores. Tradicionalmente, la gestión de la PAC y el seguimiento del historial de parcelas se realiza mediante documentos en papel, hojas de cálculo dispersas y búsquedas manuales en el visor SIGPAC, lo que supone una elevada carga administrativa y una fuente frecuente de errores.

La aplicación desarrollada centraliza toda esta información en un único sistema digital, accesible y estructurado, lo que reduce el tiempo dedicado a tareas administrativas y minimiza la probabilidad de errores en la declaración de ayudas. Esto resulta especialmente relevante para agricultores de pequeña y mediana escala, que no siempre disponen de asesoramiento técnico especializado.

Además, el proyecto surge a partir de experiencias reales del entorno familiar del autor, donde se ha podido observar de primera mano la dificultad de gestionar varias campañas agrícolas y parcelas sin herramientas digitales adecuadas. De este modo, la aplicación responde a una necesidad real del sector y contribuye a la modernización y digitalización del medio rural.

\section{Contribución a la sostenibilidad económica}

En términos económicos, el sistema desarrollado ayuda a mejorar la rentabilidad de las explotaciones agrarias. La correcta planificación de cultivos y el cumplimiento de los requisitos de la PAC permiten a los agricultores acceder a las ayudas económicas sin penalizaciones ni pérdidas derivadas de errores administrativos.

La generación de propuestas de cultivo mediante inteligencia artificial ofrece recomendaciones que tienen en cuenta tanto la productividad agronómica como los criterios normativos, lo que facilita la toma de decisiones informadas. Esto puede traducirse en una mejor asignación de cultivos a las parcelas, una optimización del uso de los recursos disponibles y, en consecuencia, una mejora de los resultados económicos de la explotación.

Asimismo, el uso de una herramienta digital reduce costes indirectos asociados al tiempo invertido en gestiones manuales, consultas externas o correcciones posteriores de errores en las declaraciones.

\section{Alineación con los Objetivos de Desarrollo Sostenible}

El proyecto se encuentra alineado con varios de los Objetivos de Desarrollo Sostenible (ODS) definidos por las Naciones Unidas, entre los que destacan:

\begin{itemize}
    \item \textbf{ODS 2: Hambre cero}, al contribuir a una agricultura más eficiente, productiva y sostenible.
    \item \textbf{ODS 9: Industria, innovación e infraestructura}, mediante el uso de tecnologías digitales e inteligencia artificial aplicadas al sector agrario.
    \item \textbf{ODS 12: Producción y consumo responsables}, fomentando prácticas agrícolas que optimizan el uso de recursos y reducen el impacto ambiental.
    \item \textbf{ODS 13: Acción por el clima}, al promover prácticas que mejoran la salud del suelo y reducen la huella ambiental de la actividad agrícola.
\end{itemize}

\section{Conclusión}

En conclusión, el presente Trabajo Fin de Grado no solo aborda un problema técnico relacionado con la gestión de datos agrícolas, sino que también aporta una solución alineada con los principios de sostenibilidad ambiental, social y económica. La aplicación desarrollada facilita el cumplimiento de la normativa vigente, mejora la eficiencia de las explotaciones agrarias y contribuye a la modernización del sector agrícola.

Este enfoque demuestra cómo el desarrollo de software puede convertirse en una herramienta clave para avanzar hacia un modelo de agricultura más sostenible, resiliente y adaptado a los retos actuales y futuros.
