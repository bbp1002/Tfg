\capitulo{5}{Documentación de usuario}
\section{Introducción}

Este apéndice describe el uso de la aplicación desde el punto de vista del usuario final. Está orientado principalmente a agricultores o técnicos agrícolas que deseen gestionar su explotación de forma digital y obtener apoyo en la toma de decisiones relacionadas con la Política Agraria Común (PAC).

No se requieren conocimientos técnicos avanzados para el uso del sistema, ya que la aplicación ha sido diseñada para ser intuitiva y accesible.

\section{Requisitos de usuario}
Para utilizar la aplicación, el usuario debe cumplir los siguientes requisitos mínimos:
\begin{itemize}
    \item Disponer de un dispositivo con acceso a Internet.
    \item Contar con un navegador web actualizado.
    \item Tener una cuenta de usuario registrada en el sistema.
    \item Disponer de los ficheros PAC oficiales en formato Excel para la importación de datos.
\end{itemize}
No es necesario que el usuario tenga conocimientos informáticos avanzados ni experiencia previa con herramientas de gestión agrícola digital.

\section{Instalación}
La aplicación se ha diseñado siguiendo un modelo cliente-servidor. El usuario no necesita realizar ninguna instalación local, ya que el acceso se realiza a través de un navegador web.

El backend se ejecuta en un servidor que expone la API, mientras que el frontend proporciona la interfaz gráfica para la interacción con el sistema.

En entornos de pruebas, el acceso a la API puede realizarse también mediante herramientas como Swagger o Postman.

\section{Manual del usuario}
\subsection{Registro y autenticación}

El primer paso para utilizar la aplicación es el registro de un usuario. Una vez registrado, el usuario puede iniciar sesión introduciendo sus credenciales.

Tras la autenticación, el sistema genera un token de seguridad que permite acceder a las funcionalidades protegidas de la aplicación.
{\imagen}{
	\begin{figure}[!h]
		\centering
		\includegraphics[width=0.9\textwidth]{img/Captura de pantalla 2026-01-14 184403.png}
		\caption{Captura del login de la aplicación}\label{fig:login}
	\end{figure}
	\FloatBarrier
}

\subsection{Gestión de parcelas y recintos}
Una vez autenticado, el usuario puede gestionar las parcelas de su explotación agrícola. Cada parcela puede contener uno o varios recintos, que representan subdivisiones oficiales del SIGPAC.

El sistema permite asignar un nombre personalizado a cada parcela, facilitando su identificación y evitando la necesidad de recordar referencias administrativas complejas.

El sistema permite asignar un nombre personalizado a cada parcela, facilitando su identificación y evitando la necesidad de recordar referencias administrativas complejas.

Esta funcionalidad surge de la necesidad detectada en explotaciones reales, donde los agricultores suelen identificar las parcelas por nombres tradicionales en lugar de códigos oficiales.

\subsection{Importación de datos PAC}
La aplicación permite importar datos oficiales de la PAC mediante archivos Excel. Este proceso automatiza una tarea que tradicionalmente se realiza de forma manual, revisando documentos en papel línea por línea.

Durante la importación, el sistema procesa los recintos, superficies y cultivos declarados, almacenándolos en la base de datos para su posterior análisis.

\subsection{Generación de propuestas de cultivo mediante inteligencia artificial}
La aplicación ofrece la posibilidad de generar propuestas de cultivo para una campaña futura mediante el uso de inteligencia artificial.

La aplicación ofrece la posibilidad de generar propuestas de cultivo para una campaña futura mediante el uso de inteligencia artificial.

El usuario puede seleccionar:
\begin{itemize}
    \item La campaña objetivo.
    \item Los ecorregímenes a los que desea optar.
    \item El listado de cultivos que desea sembrar.
\end{itemize}
{\imagen}{
	\begin{figure}[!h]
		\centering
		\includegraphics[width=0.9\textwidth]{img/Captura de pantalla 2026-01-14 185717.png}
		\caption{Captura de la pantalla para generar propuestas de cultivos}\label{fig:generar-propuestas}
	\end{figure}
	\FloatBarrier
}
A partir de esta información y del histórico de la explotación, el sistema genera una propuesta optimizada que tiene en cuenta la rotación de cultivos, la diversificación y el cumplimiento de los criterios de la PAC.

\subsection{Exportación de propuestas a Excel}
Las propuestas de cultivo pueden exportarse a formato Excel, facilitando su revisión, impresión o envío a terceros como asesores agrarios o entidades colaboradoras.

El archivo exportado incluye información detallada de cada parcela, su superficie, el cultivo recomendado y los motivos de esa recomendación.

\subsection{Integración con el visor SIGPAC}

La aplicación incluye enlaces directos al visor oficial SIGPAC, permitiendo al usuario localizar rápidamente una parcela concreta sin necesidad de introducir manualmente sus referencias.

Esta funcionalidad resulta especialmente útil en explotaciones con un elevado número de parcelas, donde recordar cada referencia administrativa resulta complejo.

{\imagen}{
	\begin{figure}[!h]
		\centering
		\includegraphics[width=0.9\textwidth]{img/Captura de pantalla 2026-01-14 202526.png}
		\caption{Captura de la pantalla para buscar una parcela en SIGPAC directamente}\label{fig:ver-sigpac}
	\end{figure}
	\FloatBarrier
}