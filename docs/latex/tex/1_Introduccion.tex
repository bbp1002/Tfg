\capitulo{1}{Introducción}


La Política Agraria Común (PAC) constituye uno de los pilares fundamentales de la Unión Europea para la regulación, sostenibilidad y apoyo económico del sector agrícola. Su correcta aplicación exige a los agricultores el cumplimiento de una serie de requisitos normativos, técnicos y administrativos que afectan directamente a la gestión de sus explotaciones. Entre estos requisitos se encuentran la declaración de parcelas, el seguimiento histórico de cultivos, el cumplimiento de prácticas agronómicas obligatorias y la adaptación a los distintos ecorregímenes establecidos en cada campaña.

En la práctica, gran parte de estos procesos se realizan de forma manual o semimanual, apoyándose en hojas de cálculo, documentación dispersa y herramientas poco integradas entre sí. Esta situación genera una carga administrativa significativa para los agricultores, especialmente en explotaciones pequeñas y medianas, donde la gestión suele recaer directamente en el propio agricultor o en su entorno familiar. La complejidad normativa de la PAC, unida a la necesidad de mantener registros históricos precisos, aumenta el riesgo de errores, incumplimientos involuntarios y pérdida de ayudas económicas.

La motivación principal para la realización de este proyecto surge de una experiencia personal directa con el sector agrícola. Varios familiares cercanos desarrollan su actividad profesional como agricultores y, de manera recurrente, se enfrentan a la dificultad de gestionar toda la información requerida para la PAC de forma manual. Estas tareas, además de consumir una gran cantidad de tiempo, requieren conocimientos técnicos específicos y una interpretación constante de los criterios normativos, lo que provoca incertidumbre y dependencia de asesorías externas.

Ante esta problemática, se plantea el desarrollo de una aplicación software orientada a facilitar la gestión de explotaciones agrícolas, centralizando la información de parcelas y recintos, automatizando la importación de datos oficiales procedentes del SIGPAC y proporcionando herramientas de apoyo a la toma de decisiones. El objetivo es reducir la carga administrativa del agricultor y ofrecer una solución práctica, accesible y adaptada a la realidad del sector.

Asimismo, el proyecto incorpora técnicas de inteligencia artificial con el fin de generar propuestas de cultivo para campañas futuras. Estas propuestas tienen en cuenta el histórico de cultivos de cada parcela, los criterios de rotación y diversificación exigidos por la PAC y los ecorregímenes seleccionados por el propio agricultor. De este modo, la aplicación no solo actúa como una herramienta de gestión, sino también como un sistema de apoyo que ayuda a planificar la explotación de manera más eficiente y alineada con los requisitos normativos.

Desde el punto de vista académico, este Trabajo de Fin de Grado permite aplicar conocimientos adquiridos a lo largo de la titulación en áreas como el desarrollo de aplicaciones web, el diseño de APIs REST, la persistencia de datos mediante bases de datos relacionales, la seguridad mediante autenticación basada en tokens y la integración de servicios externos. Además, supone una oportunidad para explorar el uso práctico de modelos de inteligencia artificial generativa en un contexto real y con impacto directo en un sector tradicionalmente poco digitalizado.

\section{Estructura de la memoria}\label{estructura-de-la-memoria}

