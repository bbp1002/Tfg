\capitulo{6}{Trabajos relacionados}

\section{Introducción a los trabajos relacionados}

En este apartado se analizan distintas aplicaciones, plataformas y herramientas existentes que abordan, de forma parcial o total, problemas similares a los tratados en el presente proyecto. El objetivo no es únicamente enumerar soluciones existentes, sino contextualizar el sistema desarrollado dentro del panorama actual de herramientas digitales aplicadas al sector agrícola.

El análisis de trabajos relacionados permite identificar:
\begin{itemize}
    \item Qué problemas ya están siendo abordados.
    \item Qué limitaciones presentan las soluciones actuales.
    \item Qué aspectos innovadores aporta el proyecto desarrollado.
    \item En qué medida se diferencia de herramientas comerciales o institucionales.
\end{itemize}

\section{Aplicaciones oficiales relacionadas con la PAC}
\subsection{SIGPAC (Sistema de Información Geográfica de Parcelas Agrícolas)}

El SIGPAC es una herramienta oficial utilizada para la identificación geográfica de parcelas agrícolas. Permite consultar información como:
\begin{itemize}
    \item Localización de parcelas y recintos.
    \item Superficie declarable.
    \item Uso SIGPAC.
    \item Límites geográficos oficiales.
\end{itemize}

Sin embargo, el SIGPAC presenta varias limitaciones desde el punto de vista del agricultor:
\begin{itemize}
    \item No almacena histórico de cultivos.
    \item No permite planificación de campañas futuras.
    \item No ofrece ayuda en la toma de decisiones.
    \item Requiere búsquedas manuales parcela a parcela.
\end{itemize}

El proyecto desarrollado no sustituye al SIGPAC, sino que lo complementa, facilitando el acceso directo a cada parcela mediante enlaces automáticos y eliminando la necesidad de búsquedas repetitivas.

\subsection{Sistemas autonómicos de gestión PAC}
Muchas comunidades autónomas disponen de plataformas propias para la gestión de la solicitud PAC. 
Estas herramientas suelen centrarse en:
\begin{itemize}
    \item La tramitación administrativa.
    \item La validación de datos declarados.
    \item La comunicación con la administración.
\end{itemize}

No obstante, estas plataformas presentan carencias importantes:
\begin{itemize}
    \item Interfaces complejas.
    \item Escasa orientación al agricultor.
    \item Falta de visión global de la explotación.
    \item Ausencia de planificación agronómica.
\end{itemize}

Frente a ello, la API desarrollada pone el foco en la gestión previa a la solicitud, ayudando al agricultor a planificar correctamente antes de declarar.

\section{Software comercial de gestión agrícola}
\subsection{Cuadernos digitales de explotación}
Existen diversas soluciones comerciales conocidas como “cuadernos de explotación digital”\cite{sativum}, cuyo objetivo es registrar actividades agrícolas como:
\begin{itemize}
    \item Tratamientos fitosanitarios.
    \item Labores agrícolas.
    \item Siembras y cosechas.
    \item Insumos utilizados.
\end{itemize}

Si bien estas herramientas suponen un avance importante, presentan varias limitaciones en relación con este proyecto:
\begin{itemize}
    \item Enfoque en el registro, no en la planificación.
    \item Escasa integración con criterios PAC complejos.
    \item Dependencia de licencias de pago.
    \item Falta de personalización para explotaciones pequeñas.
\end{itemize}

El proyecto desarrollado amplía este concepto, incorporando no solo el registro, sino también el análisis histórico y la generación de propuestas.

\subsection{Plataformas de gestión integral de explotaciones}

Algunas plataformas ofrecen soluciones integrales que incluyen mapas, gestión de parcelas y análisis productivo. No obstante, suelen estar orientadas a:
\begin{itemize}
    \item Grandes explotaciones.
    \item Agricultura de precisión.
    \item Uso intensivo de sensores y maquinaria avanzada.
\end{itemize}

Esto provoca que muchas explotaciones familiares queden fuera de su alcance por:
\begin{itemize}
    \item Coste elevado.
    \item Complejidad técnica.
    \item Necesidad de formación especializada.
\end{itemize}

El sistema propuesto se posiciona como una alternativa más accesible y adaptada a explotaciones de tamaño medio.

\section{Herramientas basadas en hojas de cálculo}

En la práctica real, una gran parte de los agricultores siguen utilizando hojas de cálculo para:
\begin{itemize}
    \item Registrar cultivos por campaña.
    \item Calcular superficies.
    \item Preparar documentación PAC.
\end{itemize}

Aunque este método es flexible, presenta importantes inconvenientes:
\begin{itemize}
    \item Falta de validación automática.
    \item Alto riesgo de errores.
    \item Dificultad para mantener históricos coherentes.
    \item Imposibilidad de aplicar reglas complejas de forma automática.
\end{itemize}

El proyecto automatiza estos procesos, manteniendo la familiaridad del formato Excel para importación y exportación, pero aportando robustez y coherencia mediante una base de datos estructurada.

\section{Uso de inteligencia artificial en agricultura}

\subsection{Sistemas de recomendación agronómica}

En los últimos años han surgido soluciones que emplean inteligencia artificial para:
\begin{itemize}
    \item Recomendar cultivos.
    \item Optimizar fertilización.
    \item Predecir rendimientos.
\end{itemize}

Sin embargo, muchas de estas soluciones se centran exclusivamente en variables productivas o climáticas, dejando de lado los requisitos normativos de la PAC.

El proyecto desarrollado introduce una novedad relevante:

la integración de criterios administrativos y normativos en el proceso de recomendación, algo poco habitual en las soluciones actuales.

\subsection{IA como apoyo, no como sustitución}

A diferencia de otros sistemas que proponen decisiones cerradas, la IA utilizada en este proyecto actúa como una herramienta de apoyo. Las propuestas generadas:
\begin{itemize}
    \item Se guardan inicialmente como borrador.
    \item Incluyen justificaciones comprensibles.
    \item Pueden ser revisadas y modificadas por el usuario.
\end{itemize}

Este enfoque resulta más adecuado para un entorno donde la experiencia del agricultor sigue siendo fundamental.

\section{Comparativa global con el proyecto desarrollado}
De forma resumida, las principales diferencias entre las soluciones existentes y el proyecto desarrollado son:
\begin{itemize}
    \item Integración del histórico de cultivos por parcela.
    \item Generación automática de propuestas para campañas futuras.
    \item Enfoque específico en el cumplimiento de la PAC.
    \item Uso de IA con criterios normativos.
    \item Accesibilidad y simplicidad para el usuario final.
    \item Eliminación de búsquedas manuales repetitivas en SIGPAC.
\end{itemize}

\section{Aportación diferencial del proyecto}
El valor diferencial del proyecto no reside en una única funcionalidad, sino en la combinación de varias características:
\begin{itemize}
    \item Digitalización de procesos tradicionalmente manuales.
    \item Centralización de la información de la explotación.
    \item Reducción de errores administrativos.
    \item Ahorro de tiempo en la planificación anual.
    \item Mejora en la comprensión de los requisitos PAC.
\end{itemize}

Esta combinación convierte la aplicación en una herramienta práctica y alineada con las necesidades reales del sector agrícola.

\section{Síntesis del capítulo}
En este capítulo se ha analizado el panorama de herramientas y aplicaciones relacionadas con la gestión agrícola y la PAC. A partir de este análisis se observa que, aunque existen múltiples soluciones parciales, pocas abordan de forma conjunta la planificación, el histórico de cultivos y el cumplimiento normativo.

El proyecto desarrollado se posiciona como una solución intermedia entre herramientas institucionales y software comercial complejo, aportando una alternativa accesible, flexible y adaptada a explotaciones reales.
