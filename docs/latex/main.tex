\documentclass[a4paper,12pt,twoside]{memoir}
\usepackage{filehook}

% Castellano
\usepackage[spanish,es-tabla]{babel}
\selectlanguage{spanish}
\usepackage[utf8]{inputenc}
\usepackage[T1]{fontenc}
\usepackage{lmodern} % Scalable font
\usepackage{microtype}
\usepackage{placeins}

\RequirePackage{booktabs}
\RequirePackage[table]{xcolor}
\RequirePackage{xtab}
\RequirePackage{multirow}

% Links
\PassOptionsToPackage{hyphens}{url}\usepackage[colorlinks]{hyperref}
\hypersetup{
	allcolors = {red}
}

% Ecuaciones
\usepackage{amsmath}

% Rutas de fichero / paquete
\newcommand{\ruta}[1]{{\sffamily #1}}

% Párrafos
\nonzeroparskip

% Huérfanas y viudas
\widowpenalty100000
\clubpenalty100000

% Imágenes

% Comando para insertar una imagen en un lugar concreto.
% Los parámetros son:
% 1 --> Ruta absoluta/relativa de la figura
% 2 --> Texto a pie de figura
% 3 --> Tamaño en tanto por uno relativo al ancho de página
\usepackage{graphicx}
\newcommand{\imagen}[3]{
	\begin{figure}[!h]
		\centering
		\includegraphics[width=#3\textwidth]{#1}
		\caption{#2}\label{fig:#1}
	\end{figure}
	\FloatBarrier
}

% Comando para insertar una imagen sin posición.
% Los parámetros son:
% 1 --> Ruta absoluta/relativa de la figura
% 2 --> Texto a pie de figura
% 3 --> Tamaño en tanto por uno relativo al ancho de página
\newcommand{\imagenflotante}[3]{
	\begin{figure}
		\centering
		\includegraphics[width=#3\textwidth]{#1}
		\caption{#2}\label{fig:#1}
	\end{figure}
}

% El comando \figura nos permite insertar figuras comodamente, y utilizando
% siempre el mismo formato. Los parametros son:
% 1 --> Porcentaje del ancho de página que ocupará la figura (de 0 a 1)
% 2 --> Fichero de la imagen
% 3 --> Texto a pie de imagen
% 4 --> Etiqueta (label) para referencias
% 5 --> Opciones que queramos pasarle al \includegraphics
% 6 --> Opciones de posicionamiento a pasarle a \begin{figure}
\newcommand{\figuraConPosicion}[6]{%
  \setlength{\anchoFloat}{#1\textwidth}%
  \addtolength{\anchoFloat}{-4\fboxsep}%
  \setlength{\anchoFigura}{\anchoFloat}%
  \begin{figure}[#6]
    \begin{center}%
      \Ovalbox{%
        \begin{minipage}{\anchoFloat}%
          \begin{center}%
            \includegraphics[width=\anchoFigura,#5]{#2}%
            \caption{#3}%
            \label{#4}%
          \end{center}%
        \end{minipage}
      }%
    \end{center}%
  \end{figure}%
}

%
% Comando para incluir imágenes en formato apaisado (sin marco).
\newcommand{\figuraApaisadaSinMarco}[5]{%
  \begin{figure}%
    \begin{center}%
    \includegraphics[angle=90,height=#1\textheight,#5]{#2}%
    \caption{#3}%
    \label{#4}%
    \end{center}%
  \end{figure}%
}
% Para las tablas
\newcommand{\otoprule}{\midrule [\heavyrulewidth]}
%
% Nuevo comando para tablas pequeñas (menos de una página).
\newcommand{\tablaSmall}[5]{%
 \begin{table}
  \begin{center}
   \rowcolors {2}{gray!35}{}
   \begin{tabular}{#2}
    \toprule
    #4
    \otoprule
    #5
    \bottomrule
   \end{tabular}
   \caption{#1}
   \label{tabla:#3}
  \end{center}
 \end{table}
}

%
% Nuevo comando para tablas pequeñas (menos de una página).
\newcommand{\tablaSmallSinColores}[5]{%
 \begin{table}[H]
  \begin{center}
   \begin{tabular}{#2}
    \toprule
    #4
    \otoprule
    #5
    \bottomrule
   \end{tabular}
   \caption{#1}
   \label{tabla:#3}
  \end{center}
 \end{table}
}

\newcommand{\tablaApaisadaSmall}[5]{%
\begin{landscape}
  \begin{table}
   \begin{center}
    \rowcolors {2}{gray!35}{}
    \begin{tabular}{#2}
     \toprule
     #4
     \otoprule
     #5
     \bottomrule
    \end{tabular}
    \caption{#1}
    \label{tabla:#3}
   \end{center}
  \end{table}
\end{landscape}
}

%
% Nuevo comando para tablas grandes con cabecera y filas alternas coloreadas en gris.
\newcommand{\tabla}[6]{%
  \begin{center}
    \tablefirsthead{
      \toprule
      #5
      \otoprule
    }
    \tablehead{
      \multicolumn{#3}{l}{\small\sl continúa desde la página anterior}\\
      \toprule
      #5
      \otoprule
    }
    \tabletail{
      \hline
      \multicolumn{#3}{r}{\small\sl continúa en la página siguiente}\\
    }
    \tablelasttail{
      \hline
    }
    \bottomcaption{#1}
    \rowcolors {2}{gray!35}{}
    \begin{xtabular}{#2}
      #6
      \bottomrule
    \end{xtabular}
    \label{tabla:#4}
  \end{center}
}

%
% Nuevo comando para tablas grandes con cabecera.
\newcommand{\tablaSinColores}[6]{%
  \begin{center}
    \tablefirsthead{
      \toprule
      #5
      \otoprule
    }
    \tablehead{
      \multicolumn{#3}{l}{\small\sl continúa desde la página anterior}\\
      \toprule
      #5
      \otoprule
    }
    \tabletail{
      \hline
      \multicolumn{#3}{r}{\small\sl continúa en la página siguiente}\\
    }
    \tablelasttail{
      \hline
    }
    \bottomcaption{#1}
    \begin{xtabular}{#2}
      #6
      \bottomrule
    \end{xtabular}
    \label{tabla:#4}
  \end{center}
}

%
% Nuevo comando para tablas grandes sin cabecera.
\newcommand{\tablaSinCabecera}[5]{%
  \begin{center}
    \tablefirsthead{
      \toprule
    }
    \tablehead{
      \multicolumn{#3}{l}{\small\sl continúa desde la página anterior}\\
      \hline
    }
    \tabletail{
      \hline
      \multicolumn{#3}{r}{\small\sl continúa en la página siguiente}\\
    }
    \tablelasttail{
      \hline
    }
    \bottomcaption{#1}
  \begin{xtabular}{#2}
    #5
   \bottomrule
  \end{xtabular}
  \label{tabla:#4}
  \end{center}
}



\definecolor{cgoLight}{HTML}{EEEEEE}
\definecolor{cgoExtralight}{HTML}{FFFFFF}

%
% Nuevo comando para tablas grandes sin cabecera.
\newcommand{\tablaSinCabeceraConBandas}[5]{%
  \begin{center}
    \tablefirsthead{
      \toprule
    }
    \tablehead{
      \multicolumn{#3}{l}{\small\sl continúa desde la página anterior}\\
      \hline
    }
    \tabletail{
      \hline
      \multicolumn{#3}{r}{\small\sl continúa en la página siguiente}\\
    }
    \tablelasttail{
      \hline
    }
    \bottomcaption{#1}
    \rowcolors[]{1}{cgoExtralight}{cgoLight}

  \begin{xtabular}{#2}
    #5
   \bottomrule
  \end{xtabular}
  \label{tabla:#4}
  \end{center}
}



\graphicspath{ {./img/} }

% Capítulos
\chapterstyle{bianchi}
\newcommand{\capitulo}[2]{
	\setcounter{chapter}{#1}
	\setcounter{section}{0}
	\setcounter{figure}{0}
	\setcounter{table}{0}
	\chapter*{\thechapter.\enskip #2}
	\addcontentsline{toc}{chapter}{\thechapter.\enskip #2}
	\markboth{#2}{#2}
}

% Apéndices
\renewcommand{\appendixname}{Apéndice}
\renewcommand*\cftappendixname{\appendixname}

\newcommand{\apendice}[1]{
	%\renewcommand{\thechapter}{A}
	\chapter{#1}
}

\renewcommand*\cftappendixname{\appendixname\ }

% Formato de portada
\makeatletter
\usepackage{xcolor}
\newcommand{\tutor}[1]{\def\@tutor{#1}}
\newcommand{\course}[1]{\def\@course{#1}}
\definecolor{cpardoBox}{HTML}{E6E6FF}
\def\maketitle{
  \null
  \thispagestyle{empty}
  % Cabecera ----------------
\noindent\includegraphics[width=\textwidth]{cabecera}\vspace{1cm}%
  \vfill
  % Título proyecto y escudo informática ----------------
  \colorbox{cpardoBox}{%
    \begin{minipage}{.8\textwidth}
      \vspace{.5cm}\Large
      \begin{center}
      \textbf{TFG del Grado en Ingeniería Informática}\vspace{.6cm}\\
      \textbf{\LARGE\@title{}}
      \end{center}
      \vspace{.2cm}
    \end{minipage}

  }%
  \hfill\begin{minipage}{.20\textwidth}
    \includegraphics[width=\textwidth]{escudoInfor}
  \end{minipage}
  \vfill
  % Datos de alumno, curso y tutores ------------------
  \begin{center}%
  {%
    \noindent\LARGE
    Presentado por \@author{}\\ 
    en Universidad de Burgos --- \@date{}\\
    Tutor: \@tutor{}\\
  }%
  \end{center}%
  \null
  \cleardoublepage
  }
\makeatother

\newcommand{\nombre}{Borja Blanco Porres} %%% cambio de comando

% Datos de portada
\title{API REST en .NET para la Gestión de Explotaciones Agrícolas y Planificación de Rotación de Cultivos}
\author{\nombre}
\tutor{Carlos Cambra Baseca}
\date{15 de enero de 2026}

\begin{document}

\maketitle


%\newpage\null\thispagestyle{empty}\newpage


%%%%%%%%%%%%%%%%%%%%%%%%%%%%%%%%%%%%%%%%%%%%%%%%%%%%%%%%%%%%%%%%%%%%%%%%%%%%%%%%%%%%%%%%
\thispagestyle{empty}


\noindent\includegraphics[width=\textwidth]{cabecera}\vspace{1cm}

\noindent D. Carlos Cambra Baseca, profesor del departamento de Digitalización, área de Ciencia de la Computación e Inteligencia Artificial.

\noindent Expone:

\noindent Que el alumno D. \nombre, con DNI 71362097C, ha realizado el Trabajo final de Grado en Ingeniería Informática titulado API REST en .NET para la Gestión de Explotaciones Agrícolas y Planificación de Rotación de Cultivos. 

\noindent Y que dicho trabajo ha sido realizado por el alumno bajo la dirección del que suscribe, en virtud de lo cual se autoriza su presentación y defensa.

\begin{center} %\large
En Burgos, 15 de enero de 2026
\end{center}

\vfill\vfill\vfill

% Author and supervisor

\vfill

% para casos con solo un tutor comentar lo anterior
% y descomentar lo siguiente
Vº. Bº. del Tutor:\\[2cm]
D. Carlos Cambra Baseca


\newpage\null\thispagestyle{empty}\newpage




\frontmatter

% Abstract en castellano
\renewcommand*\abstractname{Resumen}
\begin{abstract}
La Política Agraria Común (PAC) constituye uno de los principales pilares de apoyo económico al sector agrícola europeo, pero su correcta aplicación exige a los agricultores la gestión de una gran cantidad de información administrativa, técnica y normativa. En explotaciones agrícolas de pequeño y mediano tamaño, este proceso se realiza en muchos casos de forma manual, apoyándose en documentos en papel y consultas reiteradas a herramientas externas como el visor SIGPAC, lo que incrementa notablemente el tiempo necesario para realizar el “papeleo”.

Este Trabajo Fin de Grado tiene como objetivo el desarrollo de una aplicación software orientada a facilitar la gestión de explotaciones agrícolas y la planificación de cultivos, tomando como punto de partida la experiencia directa observada en el entorno agrícola cercano al autor. La solución propuesta permite centralizar la información relativa a parcelas y recintos, almacenar el historial de cultivos por campaña y automatizar procesos habitualmente realizados de forma manual.

La aplicación se basa en una API REST que gestiona los datos de la explotación y ofrece funcionalidades como la importación de información oficial de la PAC desde ficheros Excel, la generación de propuestas de cultivo para campañas futuras y el almacenamiento de dichas propuestas. Estas propuestas se generan mediante técnicas de inteligencia artificial, considerando tanto el historial agronómico de la explotación como los criterios establecidos por la Política Agraria Común, incluyendo las Buenas Condiciones Agrarias y Medioambientales (BCAM) y los distintos ecorregímenes disponibles.

El objetivo final del proyecto es reducir la carga administrativa del agricultor, facilitar la planificación de cultivos y ofrecer una herramienta de apoyo a la toma de decisiones que contribuya a mejorar la productividad de la explotación y el cumplimiento de la normativa vigente. 

\end{abstract}

\renewcommand*\abstractname{Descriptores}
\begin{abstract}
Agricultura digital, política agraria común, PAC, SIGPAC, gestión de explotaciones agrícolas, planificación de cultivos, inteligencia artificial, API REST, ASP.NET Core, propuestas de cultivo, ecorregímenes, BCAM, automatización administrativa, sistemas de información geográfica\ldots
\end{abstract}

\clearpage

% Abstract en inglés
\renewcommand*\abstractname{Abstract}
\begin{abstract}
The Common Agricultural Policy (CAP) is one of the main pillars of economic support for the European agricultural sector, but its correct application requires farmers to manage a large amount of administrative, technical and regulatory information. On small and medium-sized farms, this process is often carried out manually, relying on paper documents and repeated consultations with external tools such as the SIGPAC viewer, which significantly increases the time needed to complete the paperwork.

The aim of this Final Degree Project is to develop a software application designed to facilitate farm management and crop planning, based on the author's direct experience in the local agricultural environment. The proposed solution allows information on plots and enclosures to be centralised, crop history to be stored by season and processes that are usually carried out manually to be automated.

The application is based on a REST API that manages farm data and offers features such as importing official CAP information from Excel files, generating crop proposals for future campaigns, and storing these proposals. These proposals are generated using artificial intelligence techniques, taking into account both the farm's agronomic history and the criteria established by the Common Agricultural Policy, including Good Agricultural and Environmental Conditions (GAEC) and the various eco-schemes available.

The ultimate goal of the project is to reduce the administrative burden on farmers, facilitate crop planning and provide a decision-making support tool that contributes to improving farm productivity and compliance with current regulations. 
\end{abstract}

\renewcommand*\abstractname{Keywords}
\begin{abstract}
Digital agriculture, common agricultural policy, CAP, SIGPAC, farm management, crop planning, artificial intelligence, REST API, ASP.NET Core, crop proposals, eco-regimes, BCAM, administrative automation, geographic information systems..
\end{abstract}

\clearpage

% Indices
\tableofcontents

\clearpage

\listoffigures

\clearpage

\listoftables
\clearpage

\mainmatter
\capitulo{1}{Introducción}


La Política Agraria Común (PAC) constituye uno de los pilares fundamentales de la Unión Europea para la regulación, sostenibilidad y apoyo económico del sector agrícola. Su correcta aplicación exige a los agricultores el cumplimiento de una serie de requisitos normativos, técnicos y administrativos que afectan directamente a la gestión de sus explotaciones. Entre estos requisitos se encuentran la declaración de parcelas, el seguimiento histórico de cultivos, el cumplimiento de prácticas agronómicas obligatorias y la adaptación a los distintos ecorregímenes establecidos en cada campaña.

En la práctica, gran parte de estos procesos se realizan de forma manual o semimanual, apoyándose en hojas de cálculo, documentación dispersa y herramientas poco integradas entre sí. Esta situación genera una carga administrativa significativa para los agricultores, especialmente en explotaciones pequeñas y medianas, donde la gestión suele recaer directamente en el propio agricultor o en su entorno familiar. La complejidad normativa de la PAC, unida a la necesidad de mantener registros históricos precisos, aumenta el riesgo de errores, incumplimientos involuntarios y pérdida de ayudas económicas.

La motivación principal para la realización de este proyecto surge de una experiencia personal directa con el sector agrícola. Varios familiares cercanos desarrollan su actividad profesional como agricultores y, de manera recurrente, se enfrentan a la dificultad de gestionar toda la información requerida para la PAC de forma manual. Estas tareas, además de consumir una gran cantidad de tiempo, requieren conocimientos técnicos específicos y una interpretación constante de los criterios normativos, lo que provoca incertidumbre y dependencia de asesorías externas.

Ante esta problemática, se plantea el desarrollo de una aplicación software orientada a facilitar la gestión de explotaciones agrícolas, centralizando la información de parcelas y recintos, automatizando la importación de datos oficiales procedentes del SIGPAC y proporcionando herramientas de apoyo a la toma de decisiones. El objetivo es reducir la carga administrativa del agricultor y ofrecer una solución práctica, accesible y adaptada a la realidad del sector.

Asimismo, el proyecto incorpora técnicas de inteligencia artificial con el fin de generar propuestas de cultivo para campañas futuras. Estas propuestas tienen en cuenta el histórico de cultivos de cada parcela, los criterios de rotación y diversificación exigidos por la PAC y los ecorregímenes seleccionados por el propio agricultor. De este modo, la aplicación no solo actúa como una herramienta de gestión, sino también como un sistema de apoyo que ayuda a planificar la explotación de manera más eficiente y alineada con los requisitos normativos.

Desde el punto de vista académico, este Trabajo de Fin de Grado permite aplicar conocimientos adquiridos a lo largo de la titulación en áreas como el desarrollo de aplicaciones web, el diseño de APIs REST, la persistencia de datos mediante bases de datos relacionales, la seguridad mediante autenticación basada en tokens y la integración de servicios externos. Además, supone una oportunidad para explorar el uso práctico de modelos de inteligencia artificial generativa en un contexto real y con impacto directo en un sector tradicionalmente poco digitalizado.

\section{Estructura de la memoria}\label{estructura-de-la-memoria}


\capitulo{2}{Objetivos del proyecto}

El objetivo principal de este proyecto es el desarrollo de una aplicación web orientada a la gestión de explotaciones agrícolas que facilite el cumplimiento de los requisitos establecidos por la Política Agraria Común (PAC). La aplicación pretende reducir la carga administrativa del agricultor mediante la automatización de tareas repetitivas, la centralización de la información y el uso de herramientas de apoyo a la toma de decisiones basadas en inteligencia artificial.

Para alcanzar este objetivo general, se definen una serie de objetivos específicos que guían el diseño, desarrollo e implementación del sistema.

\begin{table}
\centering
\caption{Resumen de los objetivos del proyecto}
\label{tab:objetivos}
\begin{tabular}{|p{4cm}|p{10cm}|}
\hline
\textbf{Tipo} & \textbf{Descripción} \\ \hline
Objetivo general & 
Desarrollar una aplicación web que permita gestionar la información agrícola de una explotación y generar propuestas de cultivo conforme a los criterios de la Política Agraria Común (PAC). \\ \hline
Objetivo específico & 
Permitir la importación automática de datos de la PAC a partir de archivos Excel oficiales. \\ \hline
Objetivo específico & 
Gestionar parcelas, recintos y su histórico de cultivos por campañas agrícolas. \\ \hline
Objetivo específico & 
Generar propuestas de cultivo mediante el uso de inteligencia artificial, teniendo en cuenta rotaciones y criterios PAC. \\ \hline
Objetivo específico & 
Facilitar la exportación de resultados y la integración con el visor SIGPAC. \\ \hline
\end{tabular}
\end{table}


\section{Objetivo general}\label{objetivo-general}
Diseñar e implementar una plataforma software que permita a los agricultores gestionar de forma eficiente sus parcelas y cultivos, integrando datos oficiales del SIGPAC, manteniendo un histórico por campañas y generando propuestas de cultivo que cumplan con la normativa PAC vigente, todo ello mediante una arquitectura segura, escalable y fácilmente extensible.

\section{Objetivos especificos}\label{objetivos-especificos}
\begin{itemize}
\tightlist
\item
  Desarrollar una API REST utilizando ASP.NET Core que sirva como núcleo del sistema, permitiendo la gestión de usuarios, parcelas, recintos, campañas y propuestas de cultivo.
\item
Implementar un sistema de importación de datos PAC desde archivos Excel, permitiendo cargar información oficial del SIGPAC de forma automatizada y evitando errores derivados de la introducción manual de datos.
\item
Diseñar un modelo de datos que permita la gestión de parcelas y recintos, incluyendo la posibilidad de asignar nombres personalizados para facilitar su identificación por parte del agricultor.
\item
Mantener un histórico de cultivos por campaña, de forma que el sistema pueda analizar la evolución de cada parcela y verificar el cumplimiento de los criterios de rotación y diversificación exigidos por la PAC.
\item
Integrar un módulo de validación del cumplimiento normativo, comprobando de manera automática los criterios asociados a las Buenas Condiciones Agrarias y Medioambientales (BCAM) y a los ecorregímenes seleccionados.
\item
Incorporar un sistema de generación de propuestas de cultivo mediante inteligencia artificial, utilizando el modelo Google Gemini, que tenga en cuenta los datos históricos, los criterios normativos y las preferencias del agricultor.
\item
Permitir el guardado de propuestas de cultivo en estado borrador, facilitando su revisión, modificación y validación antes de su aplicación definitiva.
\item
Implementar la exportación de propuestas a formato Excel, de manera que el agricultor pueda presentar o compartir fácilmente la información generada con asesorías agrarias u organismos oficiales.
\item
Proporcionar enlaces directos al visor SIGPAC, facilitando la consulta visual de las parcelas oficiales sin necesidad de abandonar la aplicación.
\item
Garantizar la seguridad del sistema mediante un mecanismo de autenticación basado en JSON Web Tokens (JWT), asegurando que únicamente los usuarios autorizados puedan acceder a sus datos.
\item
Facilitar el proceso de pruebas y validación del sistema mediante la integración de herramientas como Swagger y Postman, permitiendo documentar y comprobar el correcto funcionamiento de la API.
\item
Utilizar un sistema de control de versiones mediante GitHub \url{https://github.com/bbp1002/Tfg}, promoviendo buenas prácticas de desarrollo colaborativo y trazabilidad del código fuente.


  \end{itemize}

  \section{Objetivos técnicos y académicos}\label{objetivos-tecnicos}
  Desde el punto de vista formativo, el proyecto persigue además los siguientes objetivos:
  \begin{itemize}
\tightlist
\item
Aplicar los conocimientos adquiridos durante la titulación en el desarrollo de aplicaciones web modernas basadas en arquitecturas cliente-servidor.
\item
Profundizar en el uso de Entity Framework Core como herramienta de acceso a datos y mapeo objeto-relacional.
\item
Trabajar con bases de datos relacionales gestionadas mediante pgAdmin4.
\item
Integrar librerías externas como ClosedXML para la manipulación de archivos Excel.
\item
Explorar el uso de modelos de inteligencia artificial generativa en un contexto real y orientado a la toma de decisiones.
\item
Desarrollar una solución software alineada con un problema real del entorno profesional.
\end{itemize}
\capitulo{3}{Conceptos teóricos}

El presente proyecto se enmarca en el ámbito de la gestión de explotaciones agrícolas, un sector tradicionalmente caracterizado por una elevada carga administrativa y una fuerte dependencia de normativas públicas, especialmente en lo relativo a la Política Agraria Común (PAC). Para comprender adecuadamente el alcance y la utilidad de la aplicación desarrollada, resulta necesario introducir una serie de conceptos teóricos fundamentales relacionados con la PAC, el sistema SIGPAC y los métodos habituales de gestión de parcelas y cultivos.

Esta sección tiene como objetivo proporcionar el contexto necesario para entender la problemática abordada, así como justificar las decisiones funcionales del sistema desarrollado, sin entrar todavía en detalles técnicos de implementación, los cuales se describen en el anexo correspondiente.
\section{La Política Agraria Común (PAC)}\label{pac}
La Política Agraria Común (PAC)\cite{pac2023} es una de las políticas más relevantes de la Unión Europea y constituye un conjunto de medidas destinadas a apoyar al sector agrícola y garantizar el suministro de alimentos, la sostenibilidad medioambiental y el desarrollo rural. Desde su creación en 1962, la PAC ha evolucionado significativamente, adaptándose a los cambios económicos, sociales y ambientales del contexto europeo.
\subsection{Objetivos de la PAC}
Los objetivos principales de la PAC pueden resumirse en los siguientes puntos:
\begin{itemize}
\tightlist
\item
Garantizar una renta justa a los agricultores.
\item
Asegurar el abastecimiento estable de alimentos a la población.
\item
Promover prácticas agrícolas sostenibles y respetuosas con el medio ambiente.
\item
Favorecer el desarrollo rural y la cohesión territorial.
\item
Impulsar la modernización y digitalización del sector agrario.
\end{itemize}

En la práctica, estos objetivos se traducen en un sistema complejo de ayudas económicas, condicionado al cumplimiento de una serie de requisitos técnicos y medioambientales por parte de los agricultores.
\subsection{Complejidad administrativa de la PAC}
Uno de los principales problemas asociados a la PAC es su alta complejidad administrativa. Los agricultores deben justificar cada año:
\begin{itemize}
\tightlist
\item
Qué cultivos se han sembrado.
\item
En qué parcelas y recintos.
\item
Durante qué campañas.
\item
Bajo qué prácticas agrarias.
\item
Cumpliendo qué criterios normativos.
\end{itemize}

Esta información suele estar dispersa en documentos oficiales, archivos en papel\cite{asajaBurgos}, hojas de cálculo y plataformas digitales externas, lo que dificulta enormemente su gestión, especialmente en explotaciones familiares o de pequeño tamaño, donde no existe personal administrativo especializado.

En muchos casos, la información histórica de cultivos se conserva únicamente en documentación impresa, lo que obliga a realizar búsquedas manuales línea por línea para comprobar rotaciones, superficies o cultivos anteriores.

\subsection{Condicionalidad reforzada (BCAM)}

La PAC actual introduce el concepto de condicionalidad reforzada, que establece una serie de Buenas Condiciones Agrarias y Medioambientales (BCAM)\cite{bcam} de obligado cumplimiento para poder acceder a las ayudas.

Entre las más relevantes para este proyecto se encuentran:
\begin{itemize}
    \item 
BCAM 7: Diversificación o rotación de cultivos.
    \item 
BCAM 8: Espacios no productivos y biodiversidad (modificada en campañas recientes).

\end{itemize}

El incumplimiento de estos requisitos puede suponer la pérdida total o parcial de las ayudas, lo que convierte su correcta planificación en un aspecto crítico para los agricultores.

\section{Ecorregímenes} \label{ecorregimenes}
Los ecorregímenes son prácticas voluntarias incentivadas económicamente dentro de la PAC, orientadas a fomentar una agricultura más sostenible. A diferencia de la condicionalidad reforzada, los ecorregímenes permiten al agricultor elegir a cuáles acogerse, siempre que cumpla los requisitos establecidos.
\subsection{Principales ecorregímenes en tierras de cultivo}
Entre los ecorregímenes más relevantes para explotaciones agrícolas se encuentran:
\begin{itemize}
    \item Rotación de cultivos con especies mejorantes.
    \item Siembra directa.
    \item Espacios de biodiversidad.
\end{itemize}

Cada uno de ellos impone condiciones específicas relacionadas con porcentajes mínimos de superficie, tipos de cultivo, rotación respecto a campañas anteriores y mantenimiento del suelo.

\subsection{Dificultades en la aplicación de ecorregímenes}
La aplicación práctica de los ecorregímenes presenta varias dificultades:
\begin{itemize}
    \item Los criterios suelen expresarse en porcentajes globales de la explotación, no por parcela individual.
    \item Es necesario analizar el histórico de cultivos de varias campañas.
    \item Algunas prácticas son incompatibles entre sí.
    \item El agricultor debe anticipar decisiones de siembra futuras.
\end{itemize}
    
Estas dificultades hacen que la planificación se realice, en muchos casos, de forma aproximada o basada en la experiencia, sin herramientas que permitan simular escenarios alternativos.

\section{El sistema SIGPAC} \label{sistema-sigpac}
El Sistema de Información Geográfica de Parcelas Agrícolas (SIGPAC)\cite{sigpac} es una base de datos geográfica utilizada en España para identificar y localizar las parcelas agrícolas declaradas en la PAC.
\subsection{Estructura del SIGPAC}
Cada parcela en SIGPAC se identifica mediante una jerarquía de referencias:
\begin{itemize}
    \item Provincia
    \item Municipio
    \item Agregado
    \item Zona
    \item Polígono
    \item Parcela
    \item Recinto
\end{itemize}

Esta estructura permite localizar con precisión cualquier superficie agrícola, pero también introduce una complejidad adicional para el agricultor, que debe conocer o consultar constantemente estos datos. En la imagen~\ref{fig:sigpac} podemos ver como se hace la búsqueda en SIGPAC.
{\imagen}{
	\begin{figure}[!h]
		\centering
		\includegraphics[\textwidth]{img/Captura de pantalla 2026-01-14 070838.png}
		\caption{Captura de la búsqueda en SIGPAC}\label{fig:sigpac}
	\end{figure}
	\FloatBarrier

\subsection{Uso habitual del SIGPAC por los agricultores}
CEn la práctica, muchos agricultores no recuerdan las referencias exactas de cada parcela, especialmente cuando gestionan un elevado número de ellas. Esto obliga a:
\begin{itemize}
    \item Buscar parcelas una a una en el visor SIGPAC.
    \item Comparar mapas con documentación impresa.
    \item Perder tiempo identificando qué parcela corresponde a cada cultivo.
\end{itemize}

Este problema se agrava cuando se trabaja con documentación histórica o cuando varias parcelas tienen superficies o ubicaciones similares.

\section{Gestión tradicional de explotaciones agrícolas}
La gestión tradicional de una explotación agrícola se basa en una combinación de documentación en papel, archivos PDF oficiales de campañas anteriores, hojas de cálculo independientes y memoria personal del agricultor.

{\imagen}{
	\begin{figure}[!h]
		\centering
		\includegraphics[width=\textwidth]{img/Diagrama.jpg}
		\caption{Diagrama del flujo usado normalmente para gestionar las explotaciones}\label{fig:#1}
	\end{figure}
	\FloatBarrier
}

\subsection{Problemas del enfoque tradicional}
Este enfoque presenta múltiples inconvenientes:
\begin{itemize}
    \item Falta de trazabilidad del histórico de cultivos.
    \item Dificultad para comprobar rotaciones.
    \item Riesgo de errores en la declaración PAC.
    \item Elevado consumo de tiempo en tareas administrativas.
\end{itemize}

En el entorno familiar del autor del proyecto, estos problemas se traducen en largas jornadas revisando documentos línea por línea, comparando campañas y consultando el visor SIGPAC repetidamente para identificar parcelas concretas.

\section{Digitalización del sector agrario}
La digitalización del sector agrario se presenta como una oportunidad clave para reducir la carga administrativa y mejorar la toma de decisiones. Sin embargo, muchas soluciones existentes:
\begin{itemize}
\item Son complejas.
\item Están orientadas a grandes explotaciones.
\item Requieren conocimientos técnicos avanzados.
\end{itemize}

El proyecto desarrollado busca cubrir este vacío, proporcionando una herramienta sencilla, centralizada y adaptada a explotaciones familiares, capaz de integrar información histórica, normativa y geográfica en un único sistema.
\section{Inteligencia artificial aplicada a la agricultura}
En los últimos años, la inteligencia artificial ha comenzado a aplicarse al sector agrícola en ámbitos como la predicción de rendimientos, el análisis de suelos o la optimización del riego. Sin embargo, su uso para la planificación administrativa y normativa sigue siendo limitado.

En este proyecto, la inteligencia artificial se emplea como una herramienta de apoyo a la toma de decisiones, capaz de:
\begin{itemize}
\item Analizar históricos de cultivos.
\item Tener en cuenta criterios normativos.
\item Proponer alternativas coherentes para futuras campañas.
\end{itemize}

La IA no sustituye al agricultor, sino que actúa como un sistema de recomendación, reduciendo la complejidad del proceso y facilitando la planificación.
\capitulo{4}{Técnicas y herramientas}
El desarrollo de una aplicación software orientada a la gestión de explotaciones agrícolas y al cumplimiento de la Política Agraria Común requiere la combinación de múltiples técnicas y herramientas que permitan abordar tanto la complejidad funcional del dominio como los requisitos de calidad propios de un sistema informático moderno. En este proyecto se ha realizado una selección consciente de tecnologías que equilibran robustez, facilidad de desarrollo, mantenibilidad y valor formativo.

Este capítulo describe de forma detallada las técnicas y herramientas empleadas, así como los criterios que han motivado su elección, sin profundizar en aspectos de implementación específicos, los cuales se recogen en el anexo técnico correspondiente.

En la Tabla~\ref{tab:tecnologias} se recogen las principales tecnologías y herramientas empleadas tanto en el backend como en el frontend del sistema desarrollado.

\begin{table}[h]
\centering
\caption{Tecnologías y herramientas utilizadas en el proyecto}
\label{tab:tecnologias}
\begin{tabular}{|p{4cm}|p{3cm}|p{6cm}|}
\hline
\textbf{Tecnología} & \textbf{Tipo} & \textbf{Uso en el proyecto} \\
\hline
ASP.NET Core\cite{aspnetcore} & Framework backend & Desarrollo de la API REST y lógica de negocio \\
\hline
Entity Framework Core\cite{entityframework} & ORM & Acceso y persistencia de datos en la base de datos \\
\hline
PostgreSQL\cite{postgresql} & Base de datos & Almacenamiento de parcelas, recintos y propuestas \\
\hline
ClosedXML\cite{closedxml} & Librería & Importación y exportación de archivos Excel PAC \\
\hline
JWT\cite{jwt} & Seguridad & Autenticación y autorización de usuarios \\
\hline
React\cite{react} & Framework frontend & Desarrollo de la interfaz web para el usuario agricultor \\
\hline
Google Gemini\cite{gemini} & Inteligencia Artificial & Generación de propuestas de cultivo para campañas futuras \\
\hline
Swagger & Documentación & Pruebas y documentación interactiva de la API \\
\hline
Postman & Testing & Pruebas manuales de los endpoints REST \\
\hline
GitHub\cite{github} & Control de versiones & Almacenamiento del repositorio y control de cambios \\
\hline
Fork\cite{fork} & Cliente Git & Gestión visual de ramas y sincronización con GitHub \\
\hline
\end{tabular}
\end{table}


\section{Enfoque metodológico del desarrollo}
El proyecto se ha abordado siguiendo un enfoque incremental e iterativo, permitiendo construir el sistema de forma progresiva. Este enfoque ha resultado especialmente adecuado dada la complejidad del dominio agrícola y la necesidad de adaptar el sistema a requisitos que han ido evolucionando conforme se analizaba la normativa PAC y se obtenía retroalimentación de agricultores reales.

El desarrollo se ha estructurado en fases, comenzando por la definición del modelo de datos y las funcionalidades básicas de gestión de parcelas, para posteriormente incorporar elementos más avanzados como la validación normativa, la generación de propuestas mediante inteligencia artificial y la integración con servicios externos.

Este enfoque ha permitido:
\begin{itemize}
    \item Validar tempranamente las funcionalidades principales.
    \item Reducir el riesgo de errores de diseño.
    \item Adaptar el sistema a necesidades reales detectadas durante el desarrollo.
    \item Mantener una visión global del proyecto sin perder flexibilidad.
\end{itemize}

\section{Arquitectura cliente-servidor}
La aplicación se ha diseñado siguiendo una arquitectura cliente-servidor basada en servicios web, en la que el backend actúa como núcleo central del sistema y expone su funcionalidad mediante una API REST.

Este tipo de arquitectura resulta especialmente adecuada para aplicaciones de gestión, ya que permite desacoplar completamente la lógica de negocio de la interfaz de usuario. De este modo, el sistema puede evolucionar de forma independiente en ambos niveles, facilitando futuras ampliaciones o cambios tecnológicos.

Entre las principales ventajas de esta arquitectura destacan:
\begin{itemize}
    \item Escalabilidad del sistema.
    \item Reutilización del backend por distintos clientes.
    \item Mayor facilidad de mantenimiento.
    \item Posibilidad de integración con herramientas externas.
\end{itemize}

\section{Desarrollo del backend con ASP.NET Core}
El backend de la aplicación se ha desarrollado utilizando ASP.NET Core, un framework moderno, multiplataforma y de código abierto. Esta tecnología se ha consolidado en los últimos años como una de las principales opciones para el desarrollo de APIs REST robustas y de alto rendimiento.

La elección de ASP.NET Core responde tanto a criterios técnicos como académicos. Desde el punto de vista formativo, permite aplicar conocimientos avanzados de desarrollo backend adquiridos durante la titulación. Desde el punto de vista práctico, ofrece un entorno estable y ampliamente utilizado en entornos profesionales.

ASP.NET Core proporciona:
\begin{itemize}
    \item Soporte nativo para APIs REST.
    \item Un sistema de inyección de dependencias integrado.
    \item Gestión eficiente de peticiones HTTP.
    \item Facilidad para implementar mecanismos de seguridad.
\end{itemize}

\section{Diseño orientado a servicios y separación de responsabilidades}
Uno de los principios fundamentales aplicados en el desarrollo ha sido la separación de responsabilidades. El sistema se ha estructurado de forma que cada componente tenga una función claramente definida.

Este enfoque mejora:
\begin{itemize}
    \item La legibilidad del código.
    \item La facilidad de mantenimiento.
    \item La capacidad de realizar pruebas.
    \item La extensibilidad del sistema.
\end{itemize}
La lógica de negocio se ha aislado de los controladores, evitando que estos se conviertan en componentes excesivamente complejos y favoreciendo un diseño más limpio y organizado.

\section{Persistencia de datos y modelo relacional}
La información gestionada por el sistema presenta una estructura altamente relacional: parcelas, recintos, campañas, cultivos, propuestas y usuarios mantienen relaciones claras entre sí. Por este motivo, se ha optado por una base de datos relacional como mecanismo de persistencia.

El uso de un modelo relacional permite:
\begin{itemize}
    \item Garantizar la integridad de los datos.
    \item Representar fielmente la realidad de una explotación agrícola.
    \item Facilitar consultas complejas sobre históricos de cultivos.
    \item Mantener consistencia entre campañas agrícolas.
\end{itemize}

\section{Uso de Entity Framework Core como ORM}
Para el acceso a datos se ha empleado Entity Framework Core, un framework de mapeo objeto-relacional que permite interactuar con la base de datos utilizando entidades del dominio.

Este enfoque reduce significativamente la complejidad del código de acceso a datos y mejora la productividad del desarrollo. Además, permite centrarse en el diseño del modelo de dominio, que en este proyecto es especialmente relevante debido a la complejidad de la normativa PAC.

El uso de un ORM aporta beneficios como: abstracción del lenguaje SQL, reducción de errores, mayor coherencia entre el modelo lógico y físico y
facilidad para evolucionar el esquema de datos.

\section{Gestión de información histórica}
Uno de los aspectos más relevantes del proyecto es la gestión del histórico de cultivos por campaña. Este histórico resulta imprescindible para verificar el cumplimiento de los requisitos de rotación y diversificación exigidos por la PAC.

El sistema se ha diseñado para conservar esta información de forma estructurada, permitiendo:
\begin{itemize}
    \item Consultar cultivos de campañas anteriores.
    \item Analizar la evolución de cada parcela.
    \item Utilizar estos datos como base para la generación de propuestas futuras.
\end{itemize}

Esta capacidad supone una mejora sustancial respecto a la gestión tradicional basada en documentos en papel o archivos dispersos.

\section{Importación de datos desde Excel}

El uso de hojas de cálculo es una práctica habitual en el sector agrícola. Por este motivo, el sistema incorpora mecanismos para importar datos PAC desde archivos Excel, facilitando la integración con información oficial existente.

Esta funcionalidad permite:

\begin{itemize}
    \item Reducir la introducción manual de datos.
    \item Minimizar errores humanos.
    \item Aprovechar información histórica ya disponible.
\end{itemize}

La importación de datos se concibe como un paso clave en el proceso de digitalización de la explotación.

\section{Exportación de información en formatos estándar}
Del mismo modo, el sistema permite exportar propuestas de cultivo a formato Excel, garantizando que la información generada pueda ser compartida fácilmente con asesorías agrarias o utilizada como documentación de apoyo.

El uso de formatos estándar favorece la aceptación de la herramienta por parte del agricultor, al no imponer cambios bruscos en sus hábitos de trabajo.

\section{Seguridad y control de acceso}
La seguridad es un aspecto fundamental del sistema, ya que se gestionan datos sensibles relacionados con explotaciones agrícolas. Para garantizar la protección de esta información se ha implementado un sistema de autenticación basado en JSON Web Tokens (JWT).

Este mecanismo permite:

\begin{itemize}
    \item Identificar de forma segura a los usuarios.
    \item Restringir el acceso a información privada.
    \item Mantener la confidencialidad de los datos.
\end{itemize}

\section{Integración con servicios externos}
El proyecto integra servicios externos relevantes para el dominio agrícola, como el visor SIGPAC. Esta integración permite al agricultor acceder directamente a la representación gráfica oficial de sus parcelas sin realizar búsquedas manuales repetitivas.

Este aspecto ha sido especialmente valorado durante las entrevistas con agricultores, ya que reduce considerablemente el tiempo dedicado a tareas de identificación de parcelas.

\section{Uso de inteligencia artificial como herramienta de apoyo}
La inteligencia artificial se ha integrado como un sistema de apoyo a la toma de decisiones, no como un sustituto del agricultor. El modelo utilizado analiza datos históricos y criterios normativos para generar propuestas coherentes, reduciendo la complejidad del proceso de planificación.

Este enfoque refuerza el carácter innovador del proyecto y demuestra el potencial de la IA aplicada a problemas administrativos del sector agrario.

\section{Herramientas de documentación y validación}
El uso de herramientas como Swagger y Postman ha sido fundamental para documentar la API y validar su correcto funcionamiento. Estas herramientas permiten simular escenarios reales de uso y verificar que el sistema responde adecuadamente ante distintas situaciones.

\section{Control de versiones y buenas prácticas}
El control de versiones mediante Git y GitHub ha permitido mantener un historial claro del desarrollo, facilitando la gestión de cambios y promoviendo buenas prácticas de ingeniería del software.

\section{Digitalización del sector agrícola como contexto tecnológico}
La agricultura ha experimentado en los últimos años un proceso progresivo de digitalización, impulsado tanto por la necesidad de optimizar recursos como por el aumento de las exigencias administrativas impuestas por las políticas agrarias europeas. Sin embargo, este proceso no se ha producido de manera homogénea, especialmente en explotaciones familiares o de pequeño y mediano tamaño, donde todavía es habitual el uso de documentos en papel y hojas de cálculo no estructuradas.

En este contexto, la elección de herramientas informáticas modernas no responde únicamente a criterios técnicos, sino también a la necesidad de crear soluciones accesibles, comprensibles y adaptadas a usuarios con distintos niveles de alfabetización digital. El proyecto se sitúa precisamente en este punto de intersección entre tecnología y realidad del sector primario.

\section{Justificación de tecnologías open source}
Una decisión relevante en el desarrollo del proyecto ha sido el uso mayoritario de tecnologías de código abierto. Esta elección responde a varios factores clave:
\begin{itemize}
    \item Accesibilidad económica, especialmente relevante en el ámbito agrícola.
    \item Transparencia en el funcionamiento del software.
    \item Amplia comunidad de usuarios y desarrolladores.
    \item Abundancia de documentación y recursos formativos.
\end{itemize}

El uso de tecnologías open source garantiza que el sistema pueda ser mantenido, ampliado o adaptado en el futuro sin depender de licencias propietarias, lo cual es fundamental para la sostenibilidad del proyecto a largo plazo.

\section{Elección del lenguaje de programación y su impacto formativo}
El lenguaje C# ha sido utilizado como base del desarrollo backend. Más allá de sus características técnicas, esta elección tiene una clara justificación académica y formativa.

C# es un lenguaje fuertemente tipado, lo que favorece:
\begin{itemize}
    \item La detección temprana de errores.
    \item El diseño estructurado del software.
    \item El uso de patrones de diseño.
    \item El desarrollo de aplicaciones mantenibles.
\end{itemize}

Además, su uso en entornos profesionales convierte este proyecto en una experiencia alineada con las exigencias del mercado laboral.

\section{Gestión de complejidad normativa mediante software}

Uno de los principales retos del proyecto ha sido trasladar la complejidad de la normativa PAC a un sistema informático comprensible. Las normas relacionadas con BCAM, ecorregímenes y rotación de cultivos presentan múltiples condiciones interdependientes que resultan difíciles de gestionar manualmente.

El uso de software permite:
\begin{itemize}
    \item Sistematizar reglas complejas.
    \item Evitar interpretaciones erróneas.
    \item Aplicar criterios de forma homogénea.
    \item Reducir la carga cognitiva del agricultor.
\end{itemize}

Este aspecto justifica el uso de técnicas de validación automática y generación de propuestas asistidas por IA.

\section{Tratamiento estructurado de la información agrícola}

La información agrícola presenta características particulares:
\begin{itemize}
    \item Alta repetitividad por campañas.
    \item Dependencia temporal.
    \item Relación directa con el territorio.
    \item Importancia del contexto histórico.
\end{itemize}

Por ello, el sistema ha sido diseñado para tratar la información de manera estructurada, permitiendo consultas y análisis que serían prácticamente inviables mediante métodos manuales.

\section{Persistencia histórica como herramienta de decisión}

La persistencia del histórico de cultivos no se concibe únicamente como un mecanismo de almacenamiento, sino como una herramienta clave para la toma de decisiones. El conocimiento del pasado productivo de una parcela es fundamental para:
\begin{itemize}
    \item Planificar rotaciones.
    \item Cumplir criterios medioambientales.
    \item Optimizar la productividad.
    \item Prevenir problemas agronómicos.
\end{itemize}

Este enfoque refuerza el valor añadido del sistema frente a soluciones tradicionales.

\section{Automatización de tareas administrativas repetitivas}

Una de las principales ventajas del uso de herramientas software en el sector agrícola es la automatización de tareas administrativas que, tradicionalmente, consumen una gran cantidad de tiempo.

Entre estas tareas destacan:
\begin{itemize}
    \item Revisión de documentación PAC.
    \item Búsqueda de parcelas en SIGPAC.
    \item Cálculo de superficies por cultivo.
    \item Verificación de porcentajes exigidos.
\end{itemize}

El proyecto automatiza gran parte de estos procesos, permitiendo al agricultor centrarse en la toma de decisiones estratégicas.

\section{Integración de inteligencia artificial como sistema de apoyo}

La inteligencia artificial se ha utilizado como un sistema de apoyo, no como un sistema de decisión autónomo. Este enfoque resulta especialmente importante en un contexto donde la experiencia del agricultor sigue siendo insustituible.

La IA actúa como:
\begin{itemize}
    \item Asistente en la planificación.
    \item Herramienta de análisis de escenarios.
    \item Mecanismo de validación preliminar.
    \item Generador de propuestas justificadas.
\end{itemize}

Esta integración demuestra un uso responsable y realista de la tecnología.

\section{Diseño centrado en el usuario agrícola}

Aunque el proyecto se centra principalmente en el backend, las decisiones tecnológicas han tenido en cuenta el perfil del usuario final. El agricultor suele enfrentarse a interfaces complejas y poco intuitivas, lo que provoca rechazo hacia nuevas herramientas.

Por ello, se ha buscado:
\begin{itemize}
    \item Simplicidad en las operaciones.
    \item Uso de conceptos familiares.
    \item Minimización de pasos innecesarios.
    \item Claridad en los resultados obtenidos.
\end{itemize}

\section{Uso de formatos interoperables}

La interoperabilidad es un factor clave en la aceptación de cualquier herramienta informática. El uso de formatos como Excel permite que el sistema se integre de forma natural en el flujo de trabajo habitual del agricultor y de las asesorías agrarias.

Esto reduce la resistencia al cambio y facilita la adopción progresiva de la aplicación.

\section{Escalabilidad y evolución futura}

Las tecnologías seleccionadas permiten que el sistema pueda crecer de forma progresiva. En el futuro podrían incorporarse:
\begin{itemize}
    \item Nuevos ecorregímenes.
    \item Cambios normativos.
    \item Integración con sensores o datos climáticos.
    \item Nuevas herramientas de análisis.
\end{itemize}

La elección de herramientas modernas garantiza que estas ampliaciones sean viables.

\section{Evaluación del impacto tecnológico del proyecto}

Desde una perspectiva académica, el proyecto permite evaluar el impacto real de la tecnología en un sector tradicional. La aplicación demuestra cómo técnicas modernas de desarrollo software pueden aplicarse con éxito a problemas reales, generando un impacto tangible en la gestión diaria de una explotación agrícola.

\section{Síntesis del capítulo}

Este capítulo ha presentado de forma detallada las técnicas y herramientas empleadas, así como el contexto y la justificación de su uso. La combinación de tecnologías modernas, buenas prácticas de ingeniería del software e integración de inteligencia artificial da lugar a una solución sólida, extensible y alineada con las necesidades reales del sector agrícola.

\capitulo{5}{Aspectos relevantes del desarrollo del proyecto}

En este capítulo se analizan los aspectos más relevantes que han condicionado el desarrollo del proyecto, así como las decisiones de diseño adoptadas y los principales retos encontrados durante su implementación. El objetivo es justificar las elecciones realizadas y poner en valor el proceso seguido, más allá del resultado final.

{\imagen}{
	\begin{figure}[!h]
		\centering
		\includegraphics[width=\textwidth]{img/arquitectura_sistema.jpg}
		\caption{Arquitectura general del sistema propuesto}\label{fig:#1}
	\end{figure}
	\FloatBarrier
}

\section{Enfoque centrado en el usuario}

Uno de los aspectos más importantes del proyecto ha sido el enfoque centrado en el usuario final, en este caso el agricultor. Desde el inicio se ha tenido en cuenta que los usuarios del sistema no necesariamente poseen conocimientos técnicos avanzados, por lo que la aplicación debía ser intuitiva, clara y orientada a facilitar tareas que tradicionalmente se realizan de forma manual.

Este enfoque ha influido directamente en decisiones como el uso de formatos conocidos (por ejemplo, hojas de cálculo), la posibilidad de asignar nombres personalizados a las parcelas o la generación de propuestas de cultivo en formato borrador, permitiendo al agricultor revisarlas antes de su aplicación definitiva.

\section{Gestión de información compleja}
La gestión de la información asociada a la PAC implica trabajar con un volumen considerable de datos y con múltiples relaciones entre ellos. Parcelas, recintos, campañas agrícolas y cultivos históricos forman un conjunto de información que debe mantenerse coherente y accesible.

Uno de los retos principales ha sido estructurar estos datos de forma que permitieran su consulta y procesamiento eficiente, sin perder claridad ni consistencia. Para ello, se ha optado por un modelo de datos bien definido que permite reflejar fielmente la realidad de una explotación agrícola.

\section{Cumplimiento normativo}
El cumplimiento de la normativa de la Política Agraria Común ha sido un eje central del proyecto. Las condiciones establecidas por la PAC, como las BCAM o los ecorregímenes, introducen restricciones y porcentajes que afectan al conjunto de la explotación, no solo a parcelas individuales.

Esto ha supuesto la necesidad de analizar la explotación como un todo, teniendo en cuenta la superficie total y la distribución de cultivos. Este enfoque global ha influido en el diseño del sistema y en la forma de generar las propuestas de cultivo.

\section{Integración de inteligencia artificial}
La incorporación de inteligencia artificial ha supuesto uno de los aspectos más innovadores del proyecto. En lugar de utilizar la IA como un elemento aislado, se ha integrado como una herramienta de apoyo a la toma de decisiones.

La IA analiza los datos históricos de cultivos y los criterios normativos para generar propuestas coherentes, ayudando a reducir la complejidad del proceso. No obstante, el sistema se ha diseñado para que la decisión final siempre recaiga en el agricultor, manteniendo así el control humano sobre el resultado.

El proceso de generación de propuestas de cultivo se ha diseñado como un flujo secuencial bien definido, con el objetivo de centralizar en una única operación todos los cálculos y validaciones necesarias. Tal y como se detalla en la Tabla~\ref{tab:fases_propuesta}, el sistema recopila tanto la información introducida por el usuario como los datos históricos almacenados, construyendo un prompt estructurado que permite al modelo de inteligencia artificial generar una propuesta coherente a nivel de explotación completa. Este enfoque permite evaluar correctamente criterios porcentuales exigidos por la PAC y garantiza la trazabilidad del proceso.

\begin{table}[h]
\centering
\caption{Fases del proceso de generación de propuestas de cultivo}
\label{tab:fases_propuesta}
\begin{tabular}{|c|p{4cm}|p{7cm}|}
\hline
\textbf{Fase} & \textbf{Elemento} & \textbf{Descripción} \\ \hline
1 & Selección de parámetros & El usuario define el año de campaña, los cultivos deseados y los ecorregímenes objetivo. \\ \hline
2 & Recuperación de datos & El sistema obtiene las parcelas y el histórico de cultivos almacenados en la base de datos. \\ \hline
3 & Construcción del prompt & Se genera un prompt estructurado en formato JSON con toda la información relevante. \\ \hline
4 & Llamada a la IA & Se realiza una única llamada al modelo Google Gemini para toda la explotación. \\ \hline
5 & Validación PAC & Se comprueba el cumplimiento de criterios BCAM y ecorregímenes seleccionados. \\ \hline
6 & Almacenamiento & La propuesta se guarda en la base de datos en estado de borrador. \\ \hline
\end{tabular}
\end{table}


\section{Flexibilidad y personalización}
Otro aspecto relevante del desarrollo ha sido la necesidad de ofrecer flexibilidad al usuario. El sistema permite seleccionar los ecorregímenes a los que se desea optar, así como definir una lista de cultivos permitidos, adaptándose así a las preferencias y circunstancias de cada explotación.

Esta capacidad de personalización resulta fundamental en un entorno tan variable como el agrícola, donde las decisiones dependen de factores económicos, climáticos y personales.

\section{Interoperabilidad con herramientas externas}
La interoperabilidad ha sido un factor clave en el diseño del sistema. La integración con el visor SIGPAC mediante enlaces directos permite al agricultor consultar visualmente sus parcelas sin necesidad de introducir manualmente referencias adicionales.

Asimismo, la importación y exportación de datos en formatos estándar facilita la integración del sistema con los procedimientos administrativos habituales del sector.

\section{Seguridad y protección de datos}
El sistema gestiona información sensible relacionada con explotaciones agrícolas, por lo que la seguridad ha sido un aspecto prioritario. Se han incorporado mecanismos de autenticación y control de acceso que garantizan que cada usuario solo pueda acceder a su propia información.

Este enfoque refuerza la confianza del usuario en la aplicación y asegura un uso responsable del sistema.

\section{Desarrollo de un frontend sencillo}
Aunque el núcleo del proyecto se ha desarrollado como una API REST en ASP.NET Core, se ha implementado adicionalmente un frontend web sencillo con el objetivo de facilitar la interacción del usuario final con el sistema.

Este frontend permite:
\begin{itemize}
    \item Autenticarse en la aplicación.
    \item Consultar parcelas y recintos de la explotación.
    \item Visualizar el histórico de cultivos por campaña.
    \item Generar propuestas de cultivo mediante inteligencia artificial.
    \item Exportar dichas propuestas en formato Excel.
    \item Acceder directamente al visor SIGPAC para identificar parcelas.
\end{itemize}

La inclusión de este frontend ha permitido validar que la aplicación no solo es funcional a nivel técnico, sino también usable desde el punto de vista del agricultor, que normalmente no tiene conocimientos informáticos avanzados.

\section{Entrevistas con agricultores}

Durante el desarrollo del proyecto se han mantenido conversaciones y entrevistas informales con varios agricultores\cite{agricultores2025}, entre ellos familiares directos del autor, con el objetivo de conocer de primera mano las dificultades reales a las que se enfrentan durante la gestión de la PAC.

De estas entrevistas se desprenden problemas recurrentes como:
\begin{itemize}
    \item La necesidad de consultar numerosos documentos en papel de campañas anteriores.
    \item La dificultad para localizar rápidamente el historial de cultivos de una parcela concreta.
    \item La falta de identificación clara de las parcelas, lo que obliga a buscarlas una por una en el visor SIGPAC.
    \item El tiempo invertido en comprobar manualmente el cumplimiento de los requisitos de la PAC.
\end{itemize}
La aplicación desarrollada aborda directamente estas problemáticas, centralizando la información en una única plataforma digital y automatizando procesos que tradicionalmente se realizan de forma manual.

\section{Entrevista con técnica agrónoma}

Además de los agricultores, se ha realizado una entrevista con una técnica agrónoma\cite{ingenieraAgronoma} perteneciente a una empresa del sector, lo que ha permitido obtener una visión más profesional y técnica del problema.

Desde este punto de vista, se ha valorado especialmente:
\begin{itemize}
    \item La utilidad de disponer de un histórico digitalizado de cultivos por parcela.
    \item La generación automática de propuestas teniendo en cuenta criterios de la PAC.
    \item La posibilidad de trabajar con propuestas en estado de borrador antes de su declaración definitiva.
\end{itemize}
Asimismo, se han identificado posibles líneas de mejora, como:
\begin{itemize}
    \item Introducir más criterios técnicos en la generación de propuestas.
    \item Ampliar la base de conocimientos agronómicos de la inteligencia artificial.
    \item Mejorar la visualización de datos en el frontend.
\end{itemize}
Estas aportaciones han servido para reforzar la validez del proyecto y para definir futuras ampliaciones del sistema.


\capitulo{6}{Trabajos relacionados}

\section{Introducción a los trabajos relacionados}

En este apartado se analizan distintas aplicaciones, plataformas y herramientas existentes que abordan, de forma parcial o total, problemas similares a los tratados en el presente proyecto. El objetivo no es únicamente enumerar soluciones existentes, sino contextualizar el sistema desarrollado dentro del panorama actual de herramientas digitales aplicadas al sector agrícola.

El análisis de trabajos relacionados permite identificar:
\begin{itemize}
    \item Qué problemas ya están siendo abordados.
    \item Qué limitaciones presentan las soluciones actuales.
    \item Qué aspectos innovadores aporta el proyecto desarrollado.
    \item En qué medida se diferencia de herramientas comerciales o institucionales.
\end{itemize}

\section{Aplicaciones oficiales relacionadas con la PAC}
\subsection{SIGPAC (Sistema de Información Geográfica de Parcelas Agrícolas)}

El SIGPAC es una herramienta oficial utilizada para la identificación geográfica de parcelas agrícolas. Permite consultar información como:
\begin{itemize}
    \item Localización de parcelas y recintos.
    \item Superficie declarable.
    \item Uso SIGPAC.
    \item Límites geográficos oficiales.
\end{itemize}

Sin embargo, el SIGPAC presenta varias limitaciones desde el punto de vista del agricultor:
\begin{itemize}
    \item No almacena histórico de cultivos.
    \item No permite planificación de campañas futuras.
    \item No ofrece ayuda en la toma de decisiones.
    \item Requiere búsquedas manuales parcela a parcela.
\end{itemize}

El proyecto desarrollado no sustituye al SIGPAC, sino que lo complementa, facilitando el acceso directo a cada parcela mediante enlaces automáticos y eliminando la necesidad de búsquedas repetitivas.

\subsection{Sistemas autonómicos de gestión PAC}
Muchas comunidades autónomas disponen de plataformas propias para la gestión de la solicitud PAC. 
Estas herramientas suelen centrarse en:
\begin{itemize}
    \item La tramitación administrativa.
    \item La validación de datos declarados.
    \item La comunicación con la administración.
\end{itemize}

No obstante, estas plataformas presentan carencias importantes:
\begin{itemize}
    \item Interfaces complejas.
    \item Escasa orientación al agricultor.
    \item Falta de visión global de la explotación.
    \item Ausencia de planificación agronómica.
\end{itemize}

Frente a ello, la API desarrollada pone el foco en la gestión previa a la solicitud, ayudando al agricultor a planificar correctamente antes de declarar.

\section{Software comercial de gestión agrícola}
\subsection{Cuadernos digitales de explotación}
Existen diversas soluciones comerciales conocidas como “cuadernos de explotación digital”\cite{sativum}, cuyo objetivo es registrar actividades agrícolas como:
\begin{itemize}
    \item Tratamientos fitosanitarios.
    \item Labores agrícolas.
    \item Siembras y cosechas.
    \item Insumos utilizados.
\end{itemize}

Si bien estas herramientas suponen un avance importante, presentan varias limitaciones en relación con este proyecto:
\begin{itemize}
    \item Enfoque en el registro, no en la planificación.
    \item Escasa integración con criterios PAC complejos.
    \item Dependencia de licencias de pago.
    \item Falta de personalización para explotaciones pequeñas.
\end{itemize}

El proyecto desarrollado amplía este concepto, incorporando no solo el registro, sino también el análisis histórico y la generación de propuestas.

\subsection{Plataformas de gestión integral de explotaciones}

Algunas plataformas ofrecen soluciones integrales que incluyen mapas, gestión de parcelas y análisis productivo. No obstante, suelen estar orientadas a:
\begin{itemize}
    \item Grandes explotaciones.
    \item Agricultura de precisión.
    \item Uso intensivo de sensores y maquinaria avanzada.
\end{itemize}

Esto provoca que muchas explotaciones familiares queden fuera de su alcance por:
\begin{itemize}
    \item Coste elevado.
    \item Complejidad técnica.
    \item Necesidad de formación especializada.
\end{itemize}

El sistema propuesto se posiciona como una alternativa más accesible y adaptada a explotaciones de tamaño medio.

\section{Herramientas basadas en hojas de cálculo}

En la práctica real, una gran parte de los agricultores siguen utilizando hojas de cálculo para:
\begin{itemize}
    \item Registrar cultivos por campaña.
    \item Calcular superficies.
    \item Preparar documentación PAC.
\end{itemize}

Aunque este método es flexible, presenta importantes inconvenientes:
\begin{itemize}
    \item Falta de validación automática.
    \item Alto riesgo de errores.
    \item Dificultad para mantener históricos coherentes.
    \item Imposibilidad de aplicar reglas complejas de forma automática.
\end{itemize}

El proyecto automatiza estos procesos, manteniendo la familiaridad del formato Excel para importación y exportación, pero aportando robustez y coherencia mediante una base de datos estructurada.

\section{Uso de inteligencia artificial en agricultura}

\subsection{Sistemas de recomendación agronómica}

En los últimos años han surgido soluciones que emplean inteligencia artificial para:
\begin{itemize}
    \item Recomendar cultivos.
    \item Optimizar fertilización.
    \item Predecir rendimientos.
\end{itemize}

Sin embargo, muchas de estas soluciones se centran exclusivamente en variables productivas o climáticas, dejando de lado los requisitos normativos de la PAC.

El proyecto desarrollado introduce una novedad relevante:

la integración de criterios administrativos y normativos en el proceso de recomendación, algo poco habitual en las soluciones actuales.

\subsection{IA como apoyo, no como sustitución}

A diferencia de otros sistemas que proponen decisiones cerradas, la IA utilizada en este proyecto actúa como una herramienta de apoyo. Las propuestas generadas:
\begin{itemize}
    \item Se guardan inicialmente como borrador.
    \item Incluyen justificaciones comprensibles.
    \item Pueden ser revisadas y modificadas por el usuario.
\end{itemize}

Este enfoque resulta más adecuado para un entorno donde la experiencia del agricultor sigue siendo fundamental.

\section{Comparativa global con el proyecto desarrollado}
De forma resumida, las principales diferencias entre las soluciones existentes y el proyecto desarrollado son:
\begin{itemize}
    \item Integración del histórico de cultivos por parcela.
    \item Generación automática de propuestas para campañas futuras.
    \item Enfoque específico en el cumplimiento de la PAC.
    \item Uso de IA con criterios normativos.
    \item Accesibilidad y simplicidad para el usuario final.
    \item Eliminación de búsquedas manuales repetitivas en SIGPAC.
\end{itemize}

\section{Aportación diferencial del proyecto}
El valor diferencial del proyecto no reside en una única funcionalidad, sino en la combinación de varias características:
\begin{itemize}
    \item Digitalización de procesos tradicionalmente manuales.
    \item Centralización de la información de la explotación.
    \item Reducción de errores administrativos.
    \item Ahorro de tiempo en la planificación anual.
    \item Mejora en la comprensión de los requisitos PAC.
\end{itemize}

Esta combinación convierte la aplicación en una herramienta práctica y alineada con las necesidades reales del sector agrícola.

\section{Síntesis del capítulo}
En este capítulo se ha analizado el panorama de herramientas y aplicaciones relacionadas con la gestión agrícola y la PAC. A partir de este análisis se observa que, aunque existen múltiples soluciones parciales, pocas abordan de forma conjunta la planificación, el histórico de cultivos y el cumplimiento normativo.

El proyecto desarrollado se posiciona como una solución intermedia entre herramientas institucionales y software comercial complejo, aportando una alternativa accesible, flexible y adaptada a explotaciones reales.

\capitulo{7}{Conclusiones y Líneas de trabajo futuras}

\section{Conclusiones}
A lo largo de este Trabajo Fin de Grado se ha desarrollado una aplicación software orientada a la gestión de explotaciones agrícolas y a la planificación de cultivos, con especial atención al cumplimiento de la normativa establecida por la Política Agraria Común (PAC). El proyecto surge como respuesta a una problemática real detectada en el entorno agrícola cercano, donde la gestión de la información se realiza, en muchos casos, de forma manual, dispersa y apoyada en múltiples fuentes no integradas.

El principal objetivo del proyecto, consistente en centralizar la información de parcelas, recintos e histórico de cultivos y facilitar la generación de propuestas de siembra para campañas futuras, se ha cumplido de forma satisfactoria. La aplicación desarrollada permite importar datos oficiales de la PAC desde ficheros Excel, estructurarlos adecuadamente en una base de datos y reutilizarlos para distintos fines, evitando así la duplicación de información y reduciendo el riesgo de errores humanos.

Uno de los aspectos más relevantes del trabajo es la integración de técnicas de inteligencia artificial para la generación de propuestas de cultivo. A través de una única llamada a un modelo de IA, el sistema es capaz de analizar el historial agronómico de la explotación, la superficie total y los criterios normativos seleccionados por el usuario, generando recomendaciones coherentes y justificadas desde el punto de vista agronómico y administrativo. Este enfoque demuestra el potencial de la inteligencia artificial como herramienta de apoyo a la toma de decisiones en el sector agrícola, especialmente en un contexto normativo cada vez más complejo.

Asimismo, el sistema contempla el cumplimiento de distintos requisitos de la PAC, como las Buenas Condiciones Agrarias y Medioambientales (BCAM) y los ecorregímenes, permitiendo al usuario seleccionar aquellos a los que desea acogerse. Esto aporta flexibilidad a la aplicación y la hace adaptable a diferentes tipologías de explotación y estrategias productivas.

Desde el punto de vista técnico, el uso de una arquitectura basada en una API REST desarrollada con ASP.NET Core, junto con Entity Framework Core y una base de datos relacional, ha permitido construir un sistema modular, escalable y mantenible. La incorporación de mecanismos de autenticación mediante JWT garantiza la seguridad y el aislamiento de los datos de cada usuario, un aspecto fundamental cuando se trabaja con información sensible de carácter económico y administrativo.

Por último, el desarrollo de un frontend sencillo y la posibilidad de exportar la información a formatos estándar como Excel contribuyen a mejorar la usabilidad del sistema y facilitan su adopción por parte de usuarios con distintos niveles de competencia digital. Las valoraciones recogidas en entrevistas informales con agricultores y una técnica agrónoma ponen de manifiesto el interés práctico de la herramienta y su potencial para reducir significativamente el tiempo dedicado a tareas administrativas.

En conjunto, el proyecto demuestra que la digitalización y la aplicación de tecnologías web e inteligencia artificial pueden aportar un valor real y tangible al sector agrícola, contribuyendo a una gestión más eficiente, sostenible y alineada con la normativa vigente.

\section{Líneas de trabajo futuras}

A pesar de que los objetivos planteados inicialmente han sido alcanzados, el proyecto presenta múltiples posibilidades de ampliación y mejora que podrían abordarse en trabajos futuros o en una evolución real de la aplicación.

Una primera línea de mejora sería la ampliación del frontend, desarrollando una interfaz más completa y visual que permita al usuario gestionar de forma gráfica sus parcelas, visualizar mapas interactivos y consultar estadísticas de la explotación. La integración directa de mapas SIGPAC o servicios GIS permitiría una experiencia de usuario más intuitiva y reduciría aún más la dependencia de herramientas externas.

Otra posible mejora consiste en enriquecer el modelo de inteligencia artificial incorporando más variables agronómicas, como datos climáticos históricos, previsiones meteorológicas, características del suelo o precios de mercado de los cultivos. De este modo, las propuestas de cultivo podrían no solo cumplir con la normativa de la PAC, sino también optimizar la rentabilidad económica de la explotación.

También sería interesante permitir la comparación automática entre distintas propuestas de cultivo para una misma campaña, evaluando diferentes combinaciones de ecorregímenes y cultivos permitidos. Esto ofrecería al agricultor una visión más amplia de las alternativas disponibles y facilitaría la toma de decisiones estratégicas.

Desde el punto de vista de la gestión administrativa, una futura ampliación podría incluir la generación automática de documentación necesaria para la solicitud de ayudas, así como alertas y recordatorios sobre plazos administrativos, reduciendo aún más la carga burocrática asociada a la PAC.

En cuanto a la colaboración, el sistema podría evolucionar para permitir la gestión compartida de explotaciones entre distintos usuarios, como agricultores, técnicos agrónomos o asesores, con distintos niveles de permisos. Esto facilitaría el trabajo conjunto y mejoraría la comunicación entre las partes implicadas.

Por último, una línea de trabajo relevante sería la validación del sistema mediante pruebas piloto en explotaciones reales durante varias campañas, lo que permitiría evaluar de forma objetiva el impacto de la herramienta en la productividad, el cumplimiento normativo y la reducción del tiempo dedicado a tareas administrativas.

En conclusión, el proyecto sienta una base sólida sobre la que desarrollar una solución más completa y ambiciosa, demostrando el papel fundamental que pueden desempeñar las tecnologías digitales en la modernización y sostenibilidad del sector agrícola.


\bibliographystyle{plain}
\bibliography{bibliografia}


\end{document}
